\documentclass[a4paper,10pt]{article}
%\documentclass[a4paper,10pt]{scrartcl}

\usepackage[utf8]{inputenc}
\usepackage{hyperref}

\title{Conflict Reduction by Rulebased Postprocessing in C3PO}
\author{Lukas Rötzer, Peter Schmidt}
\date{25.4.2013}

\pdfinfo{%
  /Title    (Conflict Reduction by Rulebased Postprocessing in C3PO)
  /Author   (Lukas Rötzer, Peter Schmidt)
  /Creator  (Lukas Rötzer, Peter Schmidt)
  /Producer ()
  /Subject  ()
  /Keywords (C3PO, digital preservation, preservation planning, profiling, post processing)
}

\begin{document}
\maketitle
\clearpage

\section{Introduction}

The goal is to develop and apply post-processing methods and tools to FITS characterisation files within c3po to reduce the problem of "conflicting values" and significantly improve data quality. 

C3PO already has a way to add Post-Processing rules. There is an interface called com.petpet.c3po.adaptor.rules.PostProcessingRule.java in the c3po-core module, which can be implemented and added in com.petpet.c3po.controller.Controller.java. The rules are executed during the import of the FITS characterisation files and are applied after they are parsed.

The FITS adapter uses an Apache Commons Digester to parse the XML files and to trigger specific events based on rules\footnote{\url{http://commons.apache.org/proper/commons-digester/guide/core.html\#doc.Rules}}.

The \emph{getElements} method pushes a \emph{DigesterContext} on to the Digester-Stack and handles the metadata, which is basically the XML-Stream.
The adaptor rules are used to parse the parameters and the data from the XML file in java objects and properties of the digester context.

Afterwards, when the DigesterContext is completely built from the XML file, the \emph{postprocessing} are executed:
\begin{itemize}
\item the metadata is set
\item the collection of the db is set
\item if desired, date values can be obtained
\item at last, the postprocessing rules are called
\end{itemize}

\section{Problem Description}

FITS, by its design, uses some different tools to analyse the given data. The problem with this approach is, that some tools return different values for specific files.

For example, when a HTML file is evaluated, \emph{JHove} returns "Hypertext Markup Language" as format string, whereas \emph{Droid} returns "Extensible Hypertext Markup Language". This results in a conflicted state.

\section{Approach}

\end{document}
